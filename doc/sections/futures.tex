%%
%% Author: kyle
%% 10/31/17
%%

\section{Futures}
Multiple implementations of Futures (FutureTask in \gls{juc}) are in consideration as there is no
general Future that can catch all common \gls{user} errors, handle permissions, and easy for the user to
use.

% TODO:
\textcolor{blue}{TODO:}
Note, there is some confusing of the work \textit{task}. Must clarify when refering to
ICP Task, FutureTask, or the common word task used for runnable/callable passed in executor.

\subsection{Ideal Properties}
Note, not all properties may be used at once and some contradict each other.
\begin{enumerate}[label=Property \arabic*., itemindent=*, leftmargin=0pt, ref=\arabic*]
    \item \label{lst:futures:prop:1} Keep using \gls{juc} FutureTask without custom ICP implementation
    \item \label{lst:futures:prop:2} Implicit Task creation when passed a Runnable or Callable interface
    \item \label{lst:futures:prop:3} Explicit Task creation using factory methods defined in Task and CallableTask.
    \item \label{lst:futures:prop:4} Useful intent errors when using Futures incorrectly. Requires custom synchronizer.
    \item \label{lst:futures:prop:5} Explicit transfer of private object created in future before returned.
    \item \label{lst:futures:prop:6} Implicit transfer of project object using multiple if-statements to check whether
    permission may be changed, and if, change it to transfer.
\end{enumerate}

\subsection{Uses of Futures}
\begin{enumerate}[label=Use \arabic*., itemindent=*, leftmargin=0pt, ref=\arabic*]
    \item \label{lst:futures:use:1} Task who creates the Future is the only task that may retrieve the result (Private to task)
    \item \label{lst:futures:use:2} Use Future as a "joinable"\footnote{Wait for runnable to finish.} task. A custom synchronizer
    could have a \lstinline{join()}\footnote{Make \lstinline{ICPFuture extend Task}}
    method that returns void. Could it be the case where multiple tasks call join(), but only one
    calls get()? Impossible in current ICP version as permissions are object based.
    \item \label{lst:futures:use:3} A Future that allows multiple tasks retrieving a thread-safe or immutable object from
    \lstinline{get()}.\footnote{Emulate Scala's Lazy val. See \lstinline{applications.futures.Lazy}}
\end{enumerate}

\subsection{Common code used in examples and implementations}

% @formatter:off
    \begin{lstlisting}
// Return object from Callable
class Result {
    // Neither final or volatile to express memory semantics of Futures
    int value;

    Result(int value) {
        this.value = value;
    }
}

class ImmutableResult {
    final int value;

    ImmutableResult(int value) {
        this.value = value;
        ICP.setPermission(this, Permissions.getFrozenPermission());
    }
}
    \end{lstlisting}
    % @formatter:on

\subsection{Implementation A: CallableTask Wrapper} \label{sec:futures:sub:implementation:1}
Create a \lstinline{CallableTask} class which implements \lstinline{Callable} and pass the wrapped
user callable to \gls{juc} Executors. No special treatment of transfering the callable's return object's
permission is done. This is left up to the user.

\subsubsection*{How to Use}
% @formatter:off
    \begin{lstlisting}[caption=Wraping callable in CallableTask]
Future<Result> future = executorService.submit(CallableTask.ofThreadSafe(() -> {
    Result result = new Result(42);
    ICP.setPermission(result, Permissions.getTransferPermission());
    return result;
}));
assert future.get().value == 42;
    \end{lstlisting}
    \begin{lstlisting}[caption=Wrapping runnable in Task]
executorService.submit(Task.ofThreadSafe(() -> {
    Result result = new ImmutableResult(42);
    sharedCollection.add(result);
}));
    \end{lstlisting}
    % @formatter:on

Both examples rely on wrapping the Runnable and Callable interfaces in ICP compliant Tasks.
The benefit is the user may still use their own Executors and the wrapping of interfaces is not
hidden.

\subsubsection*{What problems may occur if used incorrectly}
Downsides are not being able to assert \gls{icptask}s or \gls{icpcallable}s sent into the Executor.
If a user forgets to wrap their interface, undefined behavior is possible.

\paragraph*{User does not set correct permission}~\\
The transfer permission allows any task to acquire rights to the object by accessing. If multiple
tasks try to access it by calling \lstinline{future.get()}, one will win, and the others will fail.
This is a error on the user's part since they are allowing a \textit{private} object to be shared
among multiple tasks. This is asserted by the transfer permission.

Either the data is transferable and private to only one task, or made immutable or thread-safe for
multiple tasks.

\paragraph*{Uses normal runnable instead of Task}~ \\
See \lstinine{applications.futures.Bad1}. User may forget to wrap their Runnable in a Task ready
runnable (\lstinline{Task.ofThreadSafe}).

\textcolor{red}{icp.core.IntentError: thread 'ForkJoinPool.commonPool-worker-1' is not a task}

\paragraph*{Uses normal callable instead of CallableTask}~\\
See \lstinline{applications.futures.Bad2}. User may forget to wrap their Callable in a
CallableTask ready callable.
\textcolor{red}{icp.core.IntentError: thread 'ForkJoinPool.commonPool-worker-1' is not a task}

\subsubsection*{Forcing a Callable to be Task}

\begin{tikzpicture}
    % User callable
    \umlemptyclass{UserCallable}
    \umlsimpleinterface[y=-2, alias=c1]{Callable}
    \umlimpl{UserCallable}{c1}

    % CallableTask
    \umlclass[x=5, template=V]{CallableTask}{
    - task : Task \\
    - box : AtomicReference
    }{
    + call() : V
    }
    \umlimpl{CallableTask}{c1}

    % Task
    \umlemptyclass[x=5, y=-3]{Task}
    \umlsimpleinterface[x=5, y=-5]{Runnable}
    \umlimpl{Task}{Runnable}
    \umlcompo[stereo=creates, pos stereo=0.5]{CallableTask}{Task}

    % Note for Task
    \umlnote[x=8, y=-4, geometry=|-|]{Task}{Task that runs the UserCallable and stores result}

    % FutureTask
    \umlclass[x=10, template=V]{FutureTask}{}{}
    \umlsimpleinterface[x=10, y=-2, alias=rf]{RunnableFuture}
    \umlimpl{FutureTask}{rf}

    % Associations
    \umluniassoc{CallableTask}{UserCallable}
    \umluniassoc{FutureTask}{CallableTask}
\end{tikzpicture}

The Callable interface used for submitted jobs in the Executor must be run by a \gls{icptask}. \gls{icptask}s
are not of Callable type and therefore a custom \gls{icpcallable} middle-man must handle the result.

The ICPCallable does not extend Task, but encapsulates it because the callable should not be join-able and the
"has-a" relationship on Runnable doesn't exist. The CallableTask implements Callable interface,
but the factory methods export the interface only.\footnote{Should it export
CallableTask? User could cast it unless made package-private. There is no other operations besides \lstinline{call}}

The implementation of ICPCallable uses a Task-runnable and AtomicReference as a "box" to pass from runnable to
data of call method. Creation through factory methods supports thread-safe and private callables
interfaces. When passing data to and from the "box", no permissions are set. It is up to the user
to set explicit permissions in their call method.

It should be noted that FutureTask when passed a Callable \textit{does not} wrap it in another layer to handle
returning the result. This is only done when passed a Runnable.

\subsubsection*{Trade-offs}
\begin{itemize}
    \item Doesn't require list of if branches checking the current permission\footnote{Not possible in current system}
    and seeing if it needs to be set to \textit{Transfer} before future returns.
    \item By forcing the user to set permission on data, they know by looking at the code how to use
    the future's data.
    \item There is no good way of warning the user they should set a permission before "exporting" data.
    This can lead to confusing IntentErrors.
    \item Allows users to keep using \gls{juc}'s FutureTasks.
\end{itemize}

\subsubsection*{Summary}
Allows user to keep using \gls{juc} Executor and FutureTask (Property \ref{lst:futures:prop:1}). Explicit
Task creation and permission setting (Properties \ref{lst:futures:prop:3} and \ref{lst:futures:prop:5}).

If future's result is used in multiple tasks and result is private, the IntentError would not be caught until
task touched result (Issues with Use case \ref{lst:futures:use:1}).

\subsection{Implementation B: Wrapping RunnableFuture in Task} \label{sec:futures:sub:implementation:2}
The Java Executor interface has one method \lstinline{execute(Runnable)} and all the other task execution
methods (callables, runnables, runnables with result) delegate to it. The original problem was not having the user's
callables run in a Task environment. A Task wraps a \gls{juc} FutureTask.

\subsubsection*{How to use}
% @formatter:off
\begin{lstlisting}[caption=Wrap RunnableFuture in Task]
RunnableFuture<Result> future = new FutureTask<>(() -> {
  Result result = new Result(42);
  ICP.setPermission(result, Permissions.getTransferPermission());
  return result;
});

executorService.execute(Task.ofThreadSafe(future));
assert future.get().getValue() == 42;
    \end{lstlisting}
    % @formatter:on

By still using \gls{juc} Executors, the user must not use the submit methods which take a Callable,
but build a FutureTask instead.
For normal runnables, \lstinline{executorService.execute(Task.ofThreadSafe(userRunnable))} is enough.


\subsubsection*{What problems may occur if used incorrectly}
All of the same problems "\nameref{sec:futures:sub:implementation:1}" may happen.

\subsubsection*{Wrapping classes}

\begin{tikzpicture}
    % User callable
    \umlemptyclass{UserCallable}
    \umlsimpleinterface[y=-2, alias=c1]{Callable}
    \umlimpl{UserCallable}{c1}

    % RunnableFuture
    \umlclass[x=4, template=V]{FutureTask}{}{}
    \umlinterface[x=4, y=-3]{RunnableFuture}{}{}
    \umlimpl{FutureTask}{RunnableFuture}

    % Task
    \umlemptyclass[x=8]{Task}
    \umlsimpleinterface[x=8, y=-2]{Runnable}
    \umlimpl{Task}{Runnable}

    % Associations
    \umluniassoc{FutureTask}{UserCallable}
    \umluniassoc{Task}{RunnableFuture}
\end{tikzpicture}

Only one level of wrapping of RunnableFuture inside a Task is required.

\subsubsection*{Trade-offs}
\begin{itemize}
    \item No checking of permissions as the user must set it.
    \item Know how the returning object must be used (Explicit permission set)
    \item Still no good way of warning users if incorrect Runnables or Callables are sent in executors
    \item \lstinline{execute(Runnable)} in Executor interface is the only method that may be used.
\end{itemize}

\subsubsection*{Summary}
Has the same properties and uses as "\nameref{sec:futures:sub:implementation:1}". This implementation is
the simplest from ICP perspective, but most troubling to the user as they may forget to wrap FutureTasks.

\subsection{Implementation C: ICPFutureTask}
Implement a ICPFutureTask that allows permission to be set on it and handles wrapping Callables and Runnables.
Having permissions on a Future allows futures to be private to the task who created it. Once again, this allows
using the same executors in \gls{juc}, but the user \textit{must} use the \lstinline{execute(Runnable)} passing
the RunnableFuture.

\subsubsection*{Actual Implementation}
The actual implementation of \lstinline{core.FutureTask} extends \lstinline{Task}. This allows not only using Task
methods join(), but the passed in callable or runnable does not have to be wrapped. Even if the user
does happen to wrap a runnable in a Task before sending it future, it still works. Same as wrapping the entire
RunnableFuture in Task before sending to execute method.

The constructors of Task allow a null task and atomic running boolean set to null internally for InitTask only
\footnote{Only for InitTask?}
or a constructor that takes a runnable. Since our FutureTask delegates to a \gls{juc} FutureTask, it must be created
before passing the runnable\footenote{RunnableFuture} in super call. For now, a static dummy runnable is always
passed in and never used.

The passed in runnable is never used due to another method on Task added: \lstinline{doRun()}. This package-private method
is called in \lstinline{Task.run} and allows overridable task execution. By default, it runs the runnable passed in
Task constructor.

Another downside is having to call \lstinline{Permissions.checkCall(this);} in the first line of all methods. This
FutureTask resides in the core package which is ignored.

\subsubsection*{Wrapping classes}
\begin{tikzpicture}
    % User callable
    \umlemptyclass{UserCallable}
    \umlsimpleinterface[y=-2, alias=c1]{Callable}
    \umlimpl{UserCallable}{c1}

    % FutureTask
    \umlclass[x=5, template=V]{FutureTask}{}{}
    \umlsimpleinterface[x=5, y=-2]{RunnableFuture}{}{}
    \umlimpl{FutureTask}{RunnableFuture}

    % Task
    \umlemptyclass[x=9]{Task}
    \umlinherit{FutureTask}{Task}
    \umlsimpleinterface[x=9, y=-2]{Runnable}
    \umlimpl{Task}{Runnable}

    % Associations
    \umluniassoc{FutureTask}{UserCallable}
\end{tikzpicture}

No extra wrapping outside of FutureTask wrapping a Callable inteface, which is required.

\paragraph*{Variant checking passed in Runnables} ~\\
It is still valid to wrap a runnable in a Task before sending it off to the FutureTask, but it only adds an
extra unneeded layer. Assertions can be added to avoid this in the constructors. It would still be possible
to wrap the FutureTask in a Task before sending it off to an executor.

\subsection*{Trade-offs}
\begin{itemize}
    \item Control over callable or runnable passed in through ICPFutureTask
    \item Setting permission on ICPFutureTask. Multiple calls to get now fail if private.
    \item Still explicit result permission required by user.
    \item \lstinline{execute(Runnable)} in Executor interface is only method that may be used.
\end{itemize}

\subsubsection*{Summary}
An improvement over past implementations as there are no extra unnecessary wrapping of Callable's, and allows
permissions to be set on ICPFutureTask. Properties \ref{lst:futures:prop:2}, \ref{lst:futures:prop:4},
\ref{lst:futures:prop:5} are satisfied. All Use cases can be used without any trouble.

\subsection{Implementation D: ICPExecutor}
Until now, executors are still ones from \gls{juc} without a custom ICP executor. This implementation adds
a new interface \lstinline{ICPExecutorService} that extends \lstinline{ExecutorService}. The interface methods
are overridden to return ICPFutureTask instead of Future. No new methods are added.

Methods which take a Callable are sent into the ICPFutureTask from "\nameref{Implementation C}" and the same
code is used. Methods taking a Runnable are either wrapped in a Task or sent
as is.

% @formatter:off
\begin{lstlisting}[caption=execute(Runnable)]
delegate.execute(command instanceof Task
? (Task) command
: Task.ofThreadSafe(command));
\end{lstlisting}
% @formatter:on

Tasks can still be passed in this method and wrapped if a non-task runnable runs a Task inside its run method. It is
up to the user to avoid this, but has no effect on the system.

\paragraph*{Variants of execute(Runnable)} ~\\
Instead of wrapping a non-task runnable, an IntentError can thrown to assert only Tasks are sent in. Another
option is adding a new execute method that only takes Tasks and the other execute(Runnable) method always throws
an IntentError.

Automatically wrapping a non-task runnable is a cleaner approach and allows users to use the \lstinline{ExecutorService}
interface.

\paragraph*{InvokeAll and InvokeAny} ~\\
These methods are not implemented and go directly to the delegated Executor. May not be trivial
implementations\footnote{http://grepcode.com/file/repository.grepcode.com/java/root/jdk/openjdk/8u40-b25/java/util/concurrent/ExecutorService.java}

One idea is to take the \lstinline{Collectoin<? extends Callable>} parameter in \lstinline{invokeAll}
and \lstinline{invokeAny} methods and wrap them in a CallableTask from "\nameref{sec:futures:sub:implementation:2}
Collection. This is not implemented and awaiting discussion.

\subsubsection*{How to use}
% @formatter:off
\begin{lstlisting}
// Wrap j.u.c executor in ICP executor.
ICPExecutorService executor = ICPExecutors.newICPExecutorService(
    Executors.newCachedThreadPool());

// Callable
FutureTask<Result> future = executor.submit(() -> {
  Result result = new Result(42);
  ICP.setPermission(result, Permissions.getTransferPermission());
  return result;
});
assert future.get().getValue() == 42;

// Runnable
future = executor.submit(() -> {
  // ICP permission check in runnable
  assert immutableResult.getValue() == 43;
}, new Result(42));
assert future.get().getValue() == 42;

// Runnable and using Join Permission instead of get.
Result box = new Result(0);
FutureTask<?> futureRunnable = new FutureTask<>(() -> {
  assert immutableResult.getValue() == 43; // permission check
  box.setValue(43);
}, null);
ICP.setPermission(box, futureRunnable.getJoinPermission());
executor.execute(futureRunnable);
futureRunnable.join();
assert box.getValue() == 43;
\end{lstlisting}
% @formatter:on

\subsection{Using all implemetations at once}
There are four implementations of Futures and all \textit{could} be kept in the system and left up
to the user to decide. However, the goal was to wrap code whether a runnable, callable, or future in
a task and wrapping multiple tasks inside of each other is still valid.

Consider wrapping a callable inside a ICPCallable, then passing it in the ICPFutureTask, and finally
wrapping the future in a Task to send to the ICPExecutor.

% @formatter:off
\begin{lstlisting}[caption=Multiple implemetations simultaneously]
ICPExecutorService icpexecutor = ICPExecutors.newICPExecutorService(Executors.newCachedThreadPool(
Utils.logExceptionThreadFactory()
));
future = new FutureTask<>(CallableTask.ofThreadSafe(() -> {
    Result result = new Result(42);
    ICP.setPermission(result, Permissions.getTransferPermission());
    return result;
}));

icpexecutor.execute(Task.ofThreadSafe(future));
assert future.get().getValue() == 42;
icpexecutor.shutdown();
\end{lstlisting}
% @formatter:on

\subsection{Solution}
The ICPExecutor and ICPFutureTask implementations are best suited as they allow implicit conversion to Tasks
and don't introduce any extra wrapping of Callables to Tasks or Futures to Tasks. Since ICPExecutorService
extends ExecutorService, once the user wraps \gsl{juc} Executor, it can still use the inteface without updating
their code.

InvokeAll and InvokeAny methods are not supported until further discussion. The CallableTask implementation
can be used to wrap the collection of Callable's before delegated to underlying \gsl{juc} Executor. However,
this CallableTask should be package-private and the user will have no idea, as long as it works. Avoiding
the extra wrapping is not possible.

\subsection{Applications}
Scenarios using Futures.
\textcolor{blue}{TODO: Write more applications!}

\subsubsection{Lazy evaluation}
In Scala, fields can be marked as \lstinline{lazy} and are only computed when accessed for the first time and
is thread-safe. Lazy evaluation can be implemented in Java by using Futures. See \lstinline{applications.futures.Lazy}



