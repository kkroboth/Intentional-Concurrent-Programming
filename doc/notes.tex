%%
%% Author: kyle
%% 10/27/17
%%

% Preamble
\documentclass[9pt, twoside, a4paper]{article}

% Packages
\usepackage{courier}
\usepackage{a4wide}
\usepackage{enumerate}
\usepackage[formats, final]{listings}
\usepackage[linktoc=all]{hyperref}
\usepackage{packages/tikz-uml}
\usepackage{color}
\usepackage[utf8]{inputenc}
\usepackage{enumitem}


% Glossary
\usepackage[automake, toc]{glossaries}
\makeglossary

\newglossaryentry{juc}
{
name=j.u.c,
description={The \lstinline{java.util.concurrent} package}
}

\newglossaryentry{@external}
{
name=@External,
description={Classes annotated with \lstinline{@External} annotation. Does not have permission field.}
}

\newglossaryentry{user}
{
name=User,
description={The person who uses ICP library to express their intentions on how objects are shared}
}

\newglossaryentry{icptask} {
name=ICPTask,
description={ICP Task class in \lstinline{icp.core.Task}. ICP prefix is used to not confuse reader on common word
\textit{Task} which may refer to FutureTasks, or a computational task}.
}

\newglossaryentry{icpcallable} {
name=ICPCallable,
description={Task compliant callable inteface.}
}

% End Glossary

% listings settings
\lstdefinestyle{default} {
basicstyle=\ttfamily\small,
breakatwhitespace=false,
breaklines=true,
captionpos=b,
keepspaces=true,
numbersep=5pt,
numbers=left
showspaces=false,
showstringspaces=false,
showtabs=false,
tabsize=2,
columns=spaceflexible,
}

\lstset{style=default}
\lstset{language=java}
\def\inline{\lstinline[basicstyle=\ttfamily]}

% Document
\begin{document}

    \title{ICP Notes}
    \author{Kyle Kroboth}
    \maketitle

    \paragraph*{Disclaimer}~\\
    This document is a work in progress and contains notes that may be confusing or incomplete.

    % TOC
    \setcounter{tocdepth}{4}
    \setcounter{secnumdepth}{4}
    \tableofcontents

    \section{General}

    \subsection{Constructs skipped during class editing}
    \begin{enumerate}
        \item Packages: \verb|javaassist.*|, \verb|icp.core|, \verb|sbt.|, \verb|org.testng.|,
        \verb|com.intelij.rt.|

        \item \verb|static| and \verb|abstract| methods are not instrumented.

        \item No permission checks on \verb|final| and \verb|static| fields.

        \item Permission field not added on: \verb|interfaces|, \verb|@External Annotation|
    \end{enumerate}


    \subsection{Calls to Lambdas are thread-safe}
    \lstinline{checkCall} permission checks are not inserted before lambdas. Therefor lambdas
    are thread-safe. The body of lambdas are edited; field writes and reads.

    \subsection{Synchronizers}
    Practices used when creating library synchronizers.

    \paragraph*{Final classes}
    All synchronizers should be \lstinline{final} or not extendable. For example,
    \lstinline{OneTimeLatchRegistration} has similar code in \lstinline{OneTimeLatch},
    but should not be a "is-a" relationship.

    \paragraph*{Registering methods}
    As task-based registration is becoming more relevant, code for registering multiple methods is
    repeated, and multiple Boolean TaskLocal's are used. There should be in the future a abstract
    class which handles method registration or use a single TaskLocal that keeps track of multiple
    method calls.

    \textcolor{blue}{TODO: A utility or common code for checking disjoint registrations is required.}

    \section{Ideas and Research}
    Features that are in consideration.

    \subsection{ICPCollections}
    Just how \verb|Collections.newXXX| static methods, there could be similar style for wrapping java.util
    maps, lists, and more. Other possibility is creating individual wrapper classes.

    \subsection{Use JavaAgent}
    JavaAgent's allow setting of custom class loader among other inspections. See
    https://zeroturnaround.com/rebellabs/how-to-inspect-classes-in-your-jvm/

    \subsection{Final Fields}
    Look into how JLS handles static vs non-static compiler inlines.
    See if a static call to \verb|getClass()| before
    access to final static field is made.

    \subsection{Replace Monitor locks with Reentrant instance locks}
    Without digging into the JVM, we cannot correctly check permissions for which a shared operation is accessed
    with a synchronized block or method.

    One possibility is replacing \verb|MonitorEnter| and \verb|MonitorExit| bytecode ops with lock() and unlock() methods
    of a reentrant lock. Depends on how Monitors are used.

    % @formatter:off
    \begin{lstlisting}[language=java, caption=Synchornized Method]
    public synchronized void syncMethod() {
        shared++;
    }
    \end{lstlisting}
    % @formatter:on

    % @formatter:off
    \begin{lstlisting}[language=JVMIS, caption=Bytecode for Listings 1]
public synchronized void syncMethod();
    Code:
        0: aload_0
        1: dup
        2: getfield      #2                  // Field shared:I
        5: iconst_1
        6: iadd
        7: putfield      #2                  // Field shared:I
        10: return
    \end{lstlisting}
     % @formatter:on

    In the cases of Listings 1 and 2, no bytecode ops for monitor enters and exists are used.
    Instead the JVM checks method for synchronized flag.

    % @formatter:off
    \begin{lstlisting}[language=java]
public void syncBlock() {
    synchronized(this) {
        shared++;
    }
}
    \end{lstlisting}
    % @formatter:on

    % @formatter:off
    \begin{lstlisting}[language=JVMIS]
public void syncBlock();
    Code:
       0: aload_0
       1: dup
       2: astore_1
       3: monitorenter
       4: aload_0
       5: dup
       6: getfield      #2                  // Field shared:I
       9: iconst_1
      10: iadd
      11: putfield      #2                  // Field shared:I
      14: aload_1
      15: monitorexit
      16: goto          24
      19: astore_2
      20: aload_1
      21: monitorexit
      22: aload_2
      23: athrow
      24: return
    Exception table:
       from    to  target type
           4    16    19   any
          19    22    19   any
    \end{lstlisting}
     % @formatter:on

    Explicit monitor enter and exit are used for synchronized blocks.

    One strategy for synchronized blocks is replacing a the enter and exit ops with a reentrant lock. Doing so gives
    ICP full control and allow the \gls{user} to use permissions associated with object monitors. The problem lies in how
    the system finds and replaces the bytecode.

    \subsubsection*{Possible solution 1}
    Since it may be difficult to know which object's monitor is used, inserting support functions before
    \lstinline{monitorenter} and after \lstinline{aload_0} (in these cases) may be able to get the object.
    Remove the monitor bytecode, insert reentrant lock and unlock on matching \lstinline{monitorexit}. Must use
    exception tables to always unlock lock.

    \paragraph*{Locking scenario}
    With monitor locks it is impossible\footnote{Possible through bytecode manipulation} to enter a monitor lock
    and exit in another method. Using the \gls{juc} Locks could allow one Task acquiring the lock, but having
    another task release it. This is only possible when the Thread runs two different tasks. This is avoided
    in \lstinline{SimpleReentrantLock} by setting Task owner.

    \subsubsection*{Possible solution 2}
    This was talked about on 10/30/17 and needs clarification, but instead of modifying the bytecode to replace
    monitors with ICP locks the \lstinline{Task.holdsLock()} could be used. This method can return false positives.
    \footnote{No notes why}.

    \section{Problems}

    \subsection{Use of @External Annotation}
    The External annotation is used to mark classes which should not have permission fields injected. Use cases may
    include external libraries\footnote{Right now there is no way to exclude entire or include specific packages},
    test classes and stateless static method classes.

    The problem arises when External classes are used in inheritance chains. Consider the following inheritance chain: \\
    \texttt{Object <- A <- B (External) <- C}

    Before moving on to scenarios, our class editing implementation\footnote{See PermissionSupport.addPermissionField()}
    for adding permission fields skips under these
    conditions:
    \begin{itemize}
        \item Class has \verb|@External| annotation
        \item Permission field exists in superclass
    \end{itemize}

    Whenever \texttt{ICP.setPermission(<object>, <permission>)} is called, a lookup for the permission field on
    \verb|<object>| is performed. That is done by the following:
    This algorithm returns the field closest up the inheritance chain and does not care if a class is @External.

    Finally, we have no control over how JavaAssist class loader loads classes. At any time, class A, B, or C
    can be loaded.

    \textcolor{blue}{TODO: Explain scenarios were @External annotations fail. Not that important right now.}

    \section{Synchronizers with Registrations}
    Registration refers to tasks saying \textit{I'm going to perform this operation} before calling it.
    For example, for the case of 1-time latches, a Task will inform the synchronizer it is an \textit{opener} before
    calling \lstinline{oneTimeLatch.open()}. This allows a disjoint set of opener and waiter tasks.

    As a side-effect of registration, Task-Based Permissions are possible. One-time latch permissions
    may include \textit{IsOpenPermission} and \textit{IsClosedPermission}. To use these permissions, three objects must
    be used: \textit{Data\footnote{Usually thread-safe}}, \textit{Accessor}, and \textit{Provider}. This makes coding
    awkward. See \nameref{sec:three-object-idiom} for more information.

    \subsection{Task-based permissions}
    The original intent of this project was to tie clear permissions to objects. Typically, only one invariant
    exists per Permission. \textit{IsOpen}, \textit{IsClosed}, \textit{HasLock} and so on.

    The downside is there can only be one permission for a single object. A simple workaround is having the \gls{user} create
    three classes: Underlying shared data, Accessor, and Provider. Then each object has a unique permission attached.
    For latches, \textit{IsOpen} goes to Accessor, and \textit{IsClosed} for Provider. This leaves the data object to
    be thread-safe in some cases.


    \subsubsection*{Failed Solution: Compound Permissions}
    To get around multiple permissions on a single object, a \textit{Compound} permission that uses
    composition of permissions. \lstinline{new JoinPermission(new ClosedPermission());}
    \footnote{A failed attempt of Task-Joining permission. Ideally, once a worker task joins, the master is able
    to see any shared objects the worker had (JMM). But, the worker task cannot acesses shared data after opening
    the latch while it is running.}
    Too complicated to implement and gets worse after composing more than one permission.

    \subsubsection*{Solution: Registration}
    By using Boolean TaskLocal's in the synchronizer, it is possible to have one permission \footnote{
    \textit{The Permission} to rule them all. For example, 1-time latch with registration has only
    one permission for both IsClosed and IsOpen checks.
    } tied to one object which has multiple invariants depending on the current task. The common strategy
    is:
    \begin{enumerate}
        \item Task tells synchronizer it will call an operation
        \item Task then calls operation
    \end{enumerate}

    Simple, but can be cumbersome for the \gls{user}:
    % @formatter:off
    \begin{lstlisting}
// Worker task doing some work then calling countDown on CountDownLatch
// ... work
latch.countDowner(); // Worker says it's a countdowner
latch.countDown(); // Now worker may call countDown
    \end{lstlisting}
    % @formatter:on

    The mistakes being caught are non-countdowners calling countdown operation; Master should not wait
    and call countdown. Though the example has the two methods right next to each other, it is usually
    writen as:

    % @formatter:off
    \begin{lstlisting}
CountDownLatch latch = new CountDownLatch(10); // ICP countdown latch of 10
SharedData data = new SharedData();
ICP.setPermission(data, latch.getPermission());

for(int i = 0; i < 10; i++) {
    new Thread(Task.newThreadSafeRunnable(() -> {
        // Worker registers as countdowner in first line of runnable (task)
        latch.countDowner();
        // ... more initialization

        // ... calculations

        data.set(i, <worker data>);
        latch.countDown(); // valid, worker is registered
    })).start();
}

latch.await();
// ... Access data
    \end{lstlisting}
    % @formatter:on

    The goal is having the \gls{user} register their worker tasks immediately in task initialization. It would still be
    valid if \lstinline{latch.countDowner()} was called right before the count down.

    \section{Three Object Idiom} \label{sec:three-object-idiom}
    Permissions are tied at the Object level and not method.\footnote{Previous ICP revision did method level. (confirm???)}
    Take a simple Data object that uses a AtomicInteger as its underlying data. \textit{Provider} increments the integer,
    while \textit{Accessor} retrieves the count. Using a \lstinline{CountDownLatch}, the worker tasks who countdown
    are \textit{Providers}, and the master task who awaits is the \textit{Accessor}. Ideally, there should be permissions
    that allow the worker to only call \lstinline{increment} when the latch is closed, and assert the master can only
    access through \lstinline{retrieve} method once it is open. Unfortunality, this cannot work at the method level and the
    solution involves breaking the Data up into three parts:

    \begin{tikzpicture}
        \umlemptyclass[x=2]{Data}
        \umlemptyclass[y=-2]{Accessor}
        \umlemptyclass[x=4, y=-2]{Provider}

        \umlHVassoc[name=assoc, attr=data|1, pos=1.9]{Accessor}{Data}
        \umlHVassoc[name=assoc, attr=data|1, pos=1.9]{Provider}{Data}
    \end{tikzpicture}

    % @formatter:off
    \begin{lstlisting}
public class Data {
    final AtomicInteger count = new AtomicInteger();

    public Data() {
        ICP.setPermission(this, Permissions.getPermanentlyThreadSafe());
    }

}

public class Accessor {
    final Data data;

    public Accessor(Data data) {
        this.data = data;
    }

    public int retrieve() {
        return data.count.get();
    }
}

public class Provider {
    final Data data;

    public Provider(Data data) {
        this.data = data;
    }

    public void increment() {
        this.data.count.incrementAndGet();
    }
}

// ... Somewhere after initialization and setting up synchronizer
ICP.setPermission(accessor, <permission-associated-with-master>);
ICP.setPermission(provider, <permission-associated-with-worker>);
    \end{lstlisting}
    % @formatter:on

    The problem is \lstinline{Data} object is thread safe as both the accessor and provider need it.
    This is solved with task-based permission.

    \section{One-time latch}
    Count down latches of 1. \lstinline{isOpen()} opens latch, \lstinline{await()} waits for it to
    be open. Unlike ICP CountDownLatch, open may be called multiple times because in the case of multiple
    workers, only one should open the task, but any one may nondeterministically be the opener. This includes
    having multiple openers in registration. By using an AtomicInteger to keep track of remaining finishes tasks,
    it is possible to only call \lstinline{open} once, but all have to be registered to call open.

    % @formatter:off
    \begin{lstlisting}[caption=Multiple openers, label=lst:multOpeners]
@Override
public void jobCompleted(WordResults job) {
    completeLatch.registerOpener(); // Any task may be an opener
    shared.addResult(job);
    int left = jobsLeft.decrementAndGet();
    if (left == 0) {
        // But only one task *opens* it
        completeLatch.open();
    }
}
    \end{lstlisting}
    % @formatter:on

    With only one opener, Listings \ref{lst:multOpeners}
    \footnote{See \lstinline{applications.forkjoin.synchronizers.OneTimeLatchRegistration}}
    would be impossible.

    \subsection{Without registration}
    Only has \textit{IsOpen} permission associated with synchronizer. With now registration, is is possible
    for the worker task who called \lstinline{open()} to continue accessing the shared data.\footnote{IsOpen
    permission checks if latch is open, but does not know whether the current task is a worker or master.}
    See \lstinline{applications.latches.OneWorkerOneShared} for example.

    The disadvantage is the worker may still work on the shared object after calling \lstinline{open()}. This is
    fixed with the one-time latch using registration.

    \subsection{Registration}
    Similar to the non-registration latch, but tasks must explicitly tell the synchronizer they are either a
    \textit{opener} or \textit{waiter}. This is done by calling \lstinline{registerOpener()} and
    \lstinline{registerWaiter()} methods.

    With registration, the \textit{Three Object Idiom} can be replaced with a single Data Object and single
    permission \lstinline{getPermission()}. This permission first checks the current task and has multiple
    cases depending on opener or waiter.

    See \lstinline{applications.latches.OneWorkerOneRegistration} for example.

    \section{CountDownLatch}
    The General CountDownLatch synchronizer with a positive count and operation registration. There is no
    non-registration version because of the problems with 1-time latch (non-registration).\footnote{
    Just for fun, a non-registration CountDownLatch can be implemented, but only to prove registration fixes
    the problems that workers may still access after calling countdown.
    }

    Tasks must register as a \textit{CountDowner} or \textit{Waiter} using the methods \lstinline{registerCountDowner()}
    and \lstinline{registerWaiter()}. It is a violation if a \textit{CountDowner} calls \lstinline{countdown} more than
    once or the latch is open.

    See \lstinline{applications.latches.SimpleCountDown} for example.

    %%
%% Author: kyle
%% 10/31/17
%%

\section{Futures}
Multiple implementations of Futures (FutureTask in \gls{juc}) are in consideration as there is no
general Future that can catch all common \gls{user} errors, handle permissions, and easy for the user to
use.

% TODO:
\textcolor{blue}{TODO:}
Note, there is some confusing of the work \textit{task}. Must clarify when refering to
ICP Task, FutureTask, or the common word task used for runnable/callable passed in executor.

\subsection{Ideal Properties}
Note, not all properties may be used at once and some contradict each other.
\begin{enumerate}[label=Property \arabic*., itemindent=*, leftmargin=0pt, ref=\arabic*]
    \item \label{lst:futures:prop:1} Keep using \gls{juc} FutureTask without custom ICP implementation
    \item \label{lst:futures:prop:2} Implicit Task creation when passed a Runnable or Callable interface
    \item \label{lst:futures:prop:3} Explicit Task creation using factory methods defined in Task and CallableTask.
    \item \label{lst:futures:prop:4} Useful intent errors when using Futures incorrectly. Requires custom synchronizer.
    \item \label{lst:futures:prop:5} Explicit transfer of private object created in future before returned.
    \item \label{lst:futures:prop:6} Implicit transfer of project object using multiple if-statements to check whether
    permission may be changed, and if, change it to transfer.
\end{enumerate}

\subsection{Uses of Futures}
\begin{enumerate}[label=Use \arabic*., itemindent=*, leftmargin=0pt, ref=\arabic*]
    \item \label{lst:futures:use:1} Task who creates the Future is the only task that may retrieve the result (Private to task)
    \item \label{lst:futures:use:2} Use Future as a "joinable"\footnote{Wait for runnable to finish.} task. A custom synchronizer
    could have a \lstinline{join()}\footnote{Make \lstinline{ICPFuture extend Task}}
    method that returns void. Could it be the case where multiple tasks call join(), but only one
    calls get()? Impossible in current ICP version as permissions are object based.
    \item \label{lst:futures:use:3} A Future that allows multiple tasks retrieving a thread-safe or immutable object from
    \lstinline{get()}.\footnote{Emulate Scala's Lazy val. See \lstinline{applications.futures.Lazy}}
\end{enumerate}

\subsection{Common code used in examples and implementations}

% @formatter:off
    \begin{lstlisting}
// Return object from Callable
class Result {
    // Neither final or volatile to express memory semantics of Futures
    int value;

    Result(int value) {
        this.value = value;
    }
}

class ImmutableResult {
    final int value;

    ImmutableResult(int value) {
        this.value = value;
        ICP.setPermission(this, Permissions.getFrozenPermission());
    }
}
    \end{lstlisting}
    % @formatter:on

\subsection{Implementation A: CallableTask Wrapper} \label{sec:futures:sub:implementation:1}
Create a \lstinline{CallableTask} class which implements \lstinline{Callable} and pass the wrapped
user callable to \gls{juc} Executors. No special treatment of transfering the callable's return object's
permission is done. This is left up to the user.

\subsubsection*{How to Use}
% @formatter:off
    \begin{lstlisting}[caption=Wraping callable in CallableTask]
Future<Result> future = executorService.submit(CallableTask.ofThreadSafe(() -> {
    Result result = new Result(42);
    ICP.setPermission(result, Permissions.getTransferPermission());
    return result;
}));
assert future.get().value == 42;
    \end{lstlisting}
    \begin{lstlisting}[caption=Wrapping runnable in Task]
executorService.submit(Task.ofThreadSafe(() -> {
    Result result = new ImmutableResult(42);
    sharedCollection.add(result);
}));
    \end{lstlisting}
    % @formatter:on

Both examples rely on wrapping the Runnable and Callable interfaces in ICP compliant Tasks.
The benefit is the user may still use their own Executors and the wrapping of interfaces is not
hidden.

\subsubsection*{What problems may occur if used incorrectly}
Downsides are not being able to assert \gls{icptask}s or \gls{icpcallable}s sent into the Executor.
If a user forgets to wrap their interface, undefined behavior is possible.

\paragraph*{User does not set correct permission}~\\
The transfer permission allows any task to acquire rights to the object by accessing. If multiple
tasks try to access it by calling \lstinline{future.get()}, one will win, and the others will fail.
This is a error on the user's part since they are allowing a \textit{private} object to be shared
among multiple tasks. This is asserted by the transfer permission.

Either the data is transferable and private to only one task, or made immutable or thread-safe for
multiple tasks.

\paragraph*{Uses normal runnable instead of Task}~ \\
See \lstinine{applications.futures.Bad1}. User may forget to wrap their Runnable in a Task ready
runnable (\lstinline{Task.ofThreadSafe}).

\textcolor{red}{icp.core.IntentError: thread 'ForkJoinPool.commonPool-worker-1' is not a task}

\paragraph*{Uses normal callable instead of CallableTask}~\\
See \lstinline{applications.futures.Bad2}. User may forget to wrap their Callable in a
CallableTask ready callable.
\textcolor{red}{icp.core.IntentError: thread 'ForkJoinPool.commonPool-worker-1' is not a task}

\subsubsection*{Forcing a Callable to be Task}

\begin{tikzpicture}
    % User callable
    \umlemptyclass{UserCallable}
    \umlsimpleinterface[y=-2, alias=c1]{Callable}
    \umlimpl{UserCallable}{c1}

    % CallableTask
    \umlclass[x=5, template=V]{CallableTask}{
    - task : Task \\
    - box : AtomicReference
    }{
    + call() : V
    }
    \umlimpl{CallableTask}{c1}

    % Task
    \umlemptyclass[x=5, y=-3]{Task}
    \umlsimpleinterface[x=5, y=-5]{Runnable}
    \umlimpl{Task}{Runnable}
    \umlcompo[stereo=creates, pos stereo=0.5]{CallableTask}{Task}

    % Note for Task
    \umlnote[x=8, y=-4, geometry=|-|]{Task}{Task that runs the UserCallable and stores result}

    % FutureTask
    \umlclass[x=10, template=V]{FutureTask}{}{}
    \umlsimpleinterface[x=10, y=-2, alias=rf]{RunnableFuture}
    \umlimpl{FutureTask}{rf}

    % Associations
    \umluniassoc{CallableTask}{UserCallable}
    \umluniassoc{FutureTask}{CallableTask}
\end{tikzpicture}

The Callable interface used for submitted jobs in the Executor must be run by a \gls{icptask}. \gls{icptask}s
are not of Callable type and therefore a custom \gls{icpcallable} middle-man must handle the result.

The ICPCallable does not extend Task, but encapsulates it because the callable should not be join-able and the
"has-a" relationship on Runnable doesn't exist. The CallableTask implements Callable interface,
but the factory methods export the interface only.\footnote{Should it export
CallableTask? User could cast it unless made package-private. There is no other operations besides \lstinline{call}}

The implementation of ICPCallable uses a Task-runnable and AtomicReference as a "box" to pass from runnable to
data of call method. Creation through factory methods supports thread-safe and private callables
interfaces. When passing data to and from the "box", no permissions are set. It is up to the user
to set explicit permissions in their call method.

It should be noted that FutureTask when passed a Callable \textit{does not} wrap it in another layer to handle
returning the result. This is only done when passed a Runnable.

\subsubsection*{Trade-offs}
\begin{itemize}
    \item Doesn't require list of if branches checking the current permission\footnote{Not possible in current system}
    and seeing if it needs to be set to \textit{Transfer} before future returns.
    \item By forcing the user to set permission on data, they know by looking at the code how to use
    the future's data.
    \item There is no good way of warning the user they should set a permission before "exporting" data.
    This can lead to confusing IntentErrors.
    \item Allows users to keep using \gls{juc}'s FutureTasks.
\end{itemize}

\subsubsection*{Summary}
Allows user to keep using \gls{juc} Executor and FutureTask (Property \ref{lst:futures:prop:1}). Explicit
Task creation and permission setting (Properties \ref{lst:futures:prop:3} and \ref{lst:futures:prop:5}).

If future's result is used in multiple tasks and result is private, the IntentError would not be caught until
task touched result (Issues with Use case \ref{lst:futures:use:1}).

\subsection{Implementation B: Wrapping RunnableFuture in Task} \label{sec:futures:sub:implementation:2}
The Java Executor interface has one method \lstinline{execute(Runnable)} and all the other task execution
methods (callables, runnables, runnables with result) delegate to it. The original problem was not having the user's
callables run in a Task environment. A Task wraps a \gls{juc} FutureTask.

\subsubsection*{How to use}
% @formatter:off
\begin{lstlisting}[caption=Wrap RunnableFuture in Task]
RunnableFuture<Result> future = new FutureTask<>(() -> {
  Result result = new Result(42);
  ICP.setPermission(result, Permissions.getTransferPermission());
  return result;
});

executorService.execute(Task.ofThreadSafe(future));
assert future.get().getValue() == 42;
    \end{lstlisting}
    % @formatter:on

By still using \gls{juc} Executors, the user must not use the submit methods which take a Callable,
but build a FutureTask instead.
For normal runnables, \lstinline{executorService.execute(Task.ofThreadSafe(userRunnable))} is enough.


\subsubsection*{What problems may occur if used incorrectly}
All of the same problems "\nameref{sec:futures:sub:implementation:1}" may happen.

\subsubsection*{Wrapping classes}

\begin{tikzpicture}
    % User callable
    \umlemptyclass{UserCallable}
    \umlsimpleinterface[y=-2, alias=c1]{Callable}
    \umlimpl{UserCallable}{c1}

    % RunnableFuture
    \umlclass[x=4, template=V]{FutureTask}{}{}
    \umlinterface[x=4, y=-3]{RunnableFuture}{}{}
    \umlimpl{FutureTask}{RunnableFuture}

    % Task
    \umlemptyclass[x=8]{Task}
    \umlsimpleinterface[x=8, y=-2]{Runnable}
    \umlimpl{Task}{Runnable}

    % Associations
    \umluniassoc{FutureTask}{UserCallable}
    \umluniassoc{Task}{RunnableFuture}
\end{tikzpicture}

Only one level of wrapping of RunnableFuture inside a Task is required.

\subsubsection*{Trade-offs}
\begin{itemize}
    \item No checking of permissions as the user must set it.
    \item Know how the returning object must be used (Explicit permission set)
    \item Still no good way of warning users if incorrect Runnables or Callables are sent in executors
    \item \lstinline{execute(Runnable)} in Executor interface is the only method that may be used.
\end{itemize}

\subsubsection*{Summary}
Has the same properties and uses as "\nameref{sec:futures:sub:implementation:1}". This implementation is
the simplest from ICP perspective, but most troubling to the user as they may forget to wrap FutureTasks.

\subsection{Implementation C: ICPFutureTask}
Implement a ICPFutureTask that allows permission to be set on it and handles wrapping Callables and Runnables.
Having permissions on a Future allows futures to be private to the task who created it. Once again, this allows
using the same executors in \gls{juc}, but the user \textit{must} use the \lstinline{execute(Runnable)} passing
the RunnableFuture.

\subsubsection*{Actual Implementation}
The actual implementation of \lstinline{core.FutureTask} extends \lstinline{Task}. This allows not only using Task
methods join(), but the passed in callable or runnable does not have to be wrapped. Even if the user
does happen to wrap a runnable in a Task before sending it future, it still works. Same as wrapping the entire
RunnableFuture in Task before sending to execute method.

The constructors of Task allow a null task and atomic running boolean set to null internally for InitTask only
\footnote{Only for InitTask?}
or a constructor that takes a runnable. Since our FutureTask delegates to a \gls{juc} FutureTask, it must be created
before passing the runnable\footenote{RunnableFuture} in super call. For now, a static dummy runnable is always
passed in and never used.

The passed in runnable is never used due to another method on Task added: \lstinline{doRun()}. This package-private method
is called in \lstinline{Task.run} and allows overridable task execution. By default, it runs the runnable passed in
Task constructor.

Another downside is having to call \lstinline{Permissions.checkCall(this);} in the first line of all methods. This
FutureTask resides in the core package which is ignored.

\subsubsection*{Wrapping classes}
\begin{tikzpicture}
    % User callable
    \umlemptyclass{UserCallable}
    \umlsimpleinterface[y=-2, alias=c1]{Callable}
    \umlimpl{UserCallable}{c1}

    % FutureTask
    \umlclass[x=5, template=V]{FutureTask}{}{}
    \umlsimpleinterface[x=5, y=-2]{RunnableFuture}{}{}
    \umlimpl{FutureTask}{RunnableFuture}

    % Task
    \umlemptyclass[x=9]{Task}
    \umlinherit{FutureTask}{Task}
    \umlsimpleinterface[x=9, y=-2]{Runnable}
    \umlimpl{Task}{Runnable}

    % Associations
    \umluniassoc{FutureTask}{UserCallable}
\end{tikzpicture}

No extra wrapping outside of FutureTask wrapping a Callable inteface, which is required.

\paragraph*{Variant checking passed in Runnables} ~\\
It is still valid to wrap a runnable in a Task before sending it off to the FutureTask, but it only adds an
extra unneeded layer. Assertions can be added to avoid this in the constructors. It would still be possible
to wrap the FutureTask in a Task before sending it off to an executor.

\subsection*{Trade-offs}
\begin{itemize}
    \item Control over callable or runnable passed in through ICPFutureTask
    \item Setting permission on ICPFutureTask. Multiple calls to get now fail if private.
    \item Still explicit result permission required by user.
    \item \lstinline{execute(Runnable)} in Executor interface is only method that may be used.
\end{itemize}

\subsubsection*{Summary}
An improvement over past implementations as there are no extra unnecessary wrapping of Callable's, and allows
permissions to be set on ICPFutureTask. Properties \ref{lst:futures:prop:2}, \ref{lst:futures:prop:4},
\ref{lst:futures:prop:5} are satisfied. All Use cases can be used without any trouble.

\subsection{Implementation D: ICPExecutor}
Until now, executors are still ones from \gls{juc} without a custom ICP executor. This implementation adds
a new interface \lstinline{ICPExecutorService} that extends \lstinline{ExecutorService}. The interface methods
are overridden to return ICPFutureTask instead of Future. No new methods are added.

Methods which take a Callable are sent into the ICPFutureTask from "\nameref{Implementation C}" and the same
code is used. Methods taking a Runnable are either wrapped in a Task or sent
as is.

% @formatter:off
\begin{lstlisting}[caption=execute(Runnable)]
delegate.execute(command instanceof Task
? (Task) command
: Task.ofThreadSafe(command));
\end{lstlisting}
% @formatter:on

Tasks can still be passed in this method and wrapped if a non-task runnable runs a Task inside its run method. It is
up to the user to avoid this, but has no effect on the system.

\paragraph*{Variants of execute(Runnable)} ~\\
Instead of wrapping a non-task runnable, an IntentError can thrown to assert only Tasks are sent in. Another
option is adding a new execute method that only takes Tasks and the other execute(Runnable) method always throws
an IntentError.

Automatically wrapping a non-task runnable is a cleaner approach and allows users to use the \lstinline{ExecutorService}
interface.

\paragraph*{InvokeAll and InvokeAny} ~\\
These methods are not implemented and go directly to the delegated Executor. May not be trivial
implementations\footnote{http://grepcode.com/file/repository.grepcode.com/java/root/jdk/openjdk/8u40-b25/java/util/concurrent/ExecutorService.java}

One idea is to take the \lstinline{Collectoin<? extends Callable>} parameter in \lstinline{invokeAll}
and \lstinline{invokeAny} methods and wrap them in a CallableTask from "\nameref{sec:futures:sub:implementation:2}
Collection. This is not implemented and awaiting discussion.

\subsubsection*{How to use}
% @formatter:off
\begin{lstlisting}
// Wrap j.u.c executor in ICP executor.
ICPExecutorService executor = ICPExecutors.newICPExecutorService(
    Executors.newCachedThreadPool());

// Callable
FutureTask<Result> future = executor.submit(() -> {
  Result result = new Result(42);
  ICP.setPermission(result, Permissions.getTransferPermission());
  return result;
});
assert future.get().getValue() == 42;

// Runnable
future = executor.submit(() -> {
  // ICP permission check in runnable
  assert immutableResult.getValue() == 43;
}, new Result(42));
assert future.get().getValue() == 42;

// Runnable and using Join Permission instead of get.
Result box = new Result(0);
FutureTask<?> futureRunnable = new FutureTask<>(() -> {
  assert immutableResult.getValue() == 43; // permission check
  box.setValue(43);
}, null);
ICP.setPermission(box, futureRunnable.getJoinPermission());
executor.execute(futureRunnable);
futureRunnable.join();
assert box.getValue() == 43;
\end{lstlisting}
% @formatter:on

\subsection{Using all implemetations at once}
There are four implementations of Futures and all \textit{could} be kept in the system and left up
to the user to decide. However, the goal was to wrap code whether a runnable, callable, or future in
a task and wrapping multiple tasks inside of each other is still valid.

Consider wrapping a callable inside a ICPCallable, then passing it in the ICPFutureTask, and finally
wrapping the future in a Task to send to the ICPExecutor.

% @formatter:off
\begin{lstlisting}[caption=Multiple implemetations simultaneously]
ICPExecutorService icpexecutor = ICPExecutors.newICPExecutorService(Executors.newCachedThreadPool(
Utils.logExceptionThreadFactory()
));
future = new FutureTask<>(CallableTask.ofThreadSafe(() -> {
    Result result = new Result(42);
    ICP.setPermission(result, Permissions.getTransferPermission());
    return result;
}));

icpexecutor.execute(Task.ofThreadSafe(future));
assert future.get().getValue() == 42;
icpexecutor.shutdown();
\end{lstlisting}
% @formatter:on

\subsection{Solution}
The ICPExecutor and ICPFutureTask implementations are best suited as they allow implicit conversion to Tasks
and don't introduce any extra wrapping of Callables to Tasks or Futures to Tasks. Since ICPExecutorService
extends ExecutorService, once the user wraps \gsl{juc} Executor, it can still use the inteface without updating
their code.

InvokeAll and InvokeAny methods are not supported until further discussion. The CallableTask implementation
can be used to wrap the collection of Callable's before delegated to underlying \gsl{juc} Executor. However,
this CallableTask should be package-private and the user will have no idea, as long as it works. Avoiding
the extra wrapping is not possible.

\subsection{Applications}
Scenarios using Futures.
\textcolor{blue}{TODO: Write more applications!}

\subsubsection{Lazy evaluation}
In Scala, fields can be marked as \lstinline{lazy} and are only computed when accessed for the first time and
is thread-safe. Lazy evaluation can be implemented in Java by using Futures. See \lstinline{applications.futures.Lazy}





    \section{Joining Tasks}
    A common practice of fork-join implementations involves spawning multiple worker threads, join on all of them,
    and access the shared data. ICP works with Tasks, and not Threads which means the following code will not work:

    % @formatter:off
    \begin{lstlisting}
// Fork
for(int i = 0; i < 10; i++) {
    threads[i] = new Thread(() -> {
        data[i] = new Result(i);
    });
}

// Join
for(int i = 0; i < 10; i++) {
    threads[i].join();
    data[i].getResult(); // worker done with data
}
    \end{lstlisting}
    % @formatter:on
    See \lstinline{applications.joins.Bad1} for complete example.

    There is a happens-before relationship of all the actions in each worker and then calling join() on same
    worker thread. However, no JMM checks are implemented in ICP so there needs to be a Join permission on result
    objects.

    This cannot be checked in the current system and the user must use permissions on objects that may be accessed
    after a given Task has joined\footnote{Better name is completed, finished, calling await()}. Every task has a
    \lstinline{join} and \lstinline{getJoinPermission} methods. Task-registration is used implicitly here as the
    current task running has permission access the object until the task has finished. The master task calls
    \lstinline{join} which registers itself.

    \paragraph*{Simple Example}
    % @formatter:off
    \begin{lstlisting}[caption=Example of Task.join()]
Task task = Task.ofThreadSafe(() -> {
  try {
    Thread.sleep(10);
  } catch (InterruptedException e) {
    e.printStackTrace();
  }

  data.counter = 1;
});
// set join permission on data
ICP.setPermission(data, task.getJoinPermission());

// throw away access to thread -- use task.join()
new Thread(task).start();

// join and get access to data
task.join();
assert data.counter == 1;
    \end{lstlisting}
    % @formatter:on

    \colortext{blue}{TODO: Mention failed TaskThread, TaskThreadGroup, and force setting of permissions which violates
    Hatcher's principle. Not used anymore.}

    \section{Semaphores}
    \paragraph*{Permissions} ~\\
    Each semaphore keeps a TaskLocal count of acquired permits for each task. The permission will assert
    that that task has acquired at least one permit.\footnote{Should at least N permits be a different
    permission or synchronizer?}.

    \subsection{Disjoint Semaphore}
    Semaphore were acquriers and releasers tasks are disjoint. A suitable scenario is a \lstinline{Bounded Buffer}.

    \subsection{Release Semaphore}
    Semaphore that requires the same Task who acquires N permits to release N permits.

    \paragraph*{Possible Implementation}
    How restrictive should \textit{the task who acquired N permits must release them} be?
    Once a calls \lstinline{acquire(N)} or \lstinline{acquire()} it must the same amount of
    N permits before calling another acquire.

    \newpage
    \printglossary

\end{document}